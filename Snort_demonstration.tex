% Generated by GrindEQ Word-to-LaTeX 
\documentclass{article} %%% use \documentstyle for old LaTeX compilers

\usepackage[english]{babel} %%% 'french', 'german', 'spanish', 'danish', etc.
\usepackage{amssymb}
\usepackage{amsmath}
\usepackage{txfonts}
\usepackage{mathdots}
\usepackage[classicReIm]{kpfonts}
\usepackage[dvips]{graphicx} %%% use 'pdftex' instead of 'dvips' for PDF output

% You can include more LaTeX packages here 


\begin{document}

%\selectlanguage{english} %%% remove comment delimiter ('%') and select language if required


\noindent 

\noindent 

\noindent 

\noindent 

\noindent Name : Hud Seidu Daannaa  ID: 6396082

\noindent Name: Hud Seidu Daannaa  

\noindent Name : Hud Seidu Daannaa  ID: 6396082

\noindent 
\section{Intrusion Detection System and Working with Snort}

\noindent \textit{\underbar{Components of an Intrusion Detection System (IDS):}}\underbar{}

\noindent \textbf{Sensors}: This collects data from various sources such as files, network packets and sends them to the analyzer.

\noindent \textbf{Analyzers}: The data is processed from the sensors and determine intrusion has occurred. \textbf{User interface}: this helps to view the output and manage the behavior.

\noindent \textit{\underbar{Network-based and host-based IDS with differences:}}\underbar{}

\noindent Network-based IDS's are used to monitor and analyze passing traffic on a network with the use of a dedicated platform, these are normally located inside the firewall or on the 

\noindent DMZ.

\noindent Host-based IDS's are used on host computer (individual), it takes advantage of system resources to detect intrusions by analyzing logs of operating systems, system activities and monitor other applications.

\noindent The differences are that Network-based IDS's are placed within the network and are specialized for monitoring and analyzing passing traffic while Host-based IDS's use a host computer's resources, logs of the computers operating systems to detect intrusions

\noindent \textit{\underbar{The difference between passive and reactive systems}:}

\noindent The difference between passive and reactive systems are, in a passive system, the IDS detects an intrusion, puts the information in a log and signals an alert. For reactive systems, the IDS respond to a suspicious activity or intrusion by logging off a user or by reconfiguring the IDS to block the flow of network traffic from the suspected source of intrusion.

\noindent 

\noindent 
\subsection{ 4.2 HOME\_NET set to  10.130.4.25  and  EXTERNAL\_NET  set to any}

\noindent Snort rule that alerts for FTP connections from IP address different from Home\_net : \textit{alert tcp \$EXTERNAL\_NET any -$>$ \$HOME\_NET 21 (msg:"foreign IP alert";}sid:1098760;\textit{)}

\noindent Snort rule that alerts for ``worm'' in content outgoing from Home\_net :

\noindent \textit{alert tcp \$HOME\_NET  any -$>$ EXTERNAL\_NET  any (content: "worm"; msg: "outgoing content worm";}sid:1032231\textit{)}

\noindent The rule written in the previous question will not raise an alert because, `Internet Worm' is not case sensitive, due to the specification of  the rule above (content: "worm"; )  

\noindent Snort rule that alerts for pings from External\_Net :

\noindent \textit{alert icmp \$EXTERNAL\_NET  any -$>$ \$HOME\_NET  any (msg: "Ping found";}sid:30987\textit{)}

\noindent \textit{alert tcp any any -$>$ 10.1.1.0/24 6000:6010 (msg: "}X Windows service traffic\textit{";) }The above rule alerts for detected tcp packets fr(that is to say, X windows trafic ) on any address on any port to 10.1.1.0/24 network on ports ranging from `6000 to 6010  and output a message `X Windows service traffic'.

\noindent 
\[4.3\] 
\textit{\underbar{How the following rules work:}}

\noindent \textit{alert tcp \$EXTERNAL\_NET any -$>$ \$HOME\_NET  any (msg:"SCAN FIN''; flags: F; reference:arachnids,27;)}

\noindent Make an alert on any tcp packet with a `F'  flag (FIN) leaving any port on an external\_net to any port on a  home\_net and output a message `SCAN FIN'  and with reference to arachmids external attack identification systems.

\noindent \textit{alert tcp \$HOME\_NET 23 -$>$ EXTERNAL\_NET any (msg: "Telnet login incorrect"; content: "Login incorrect" ; flag A+ reference:arachnids,127;)}

\noindent Alert any Telnet connection attempted from some home\_net on port 23 to an outside network(external\_net) on any port, with Tcp flag bits set to Ack and other set flag bits,the packet is to be filtered for a `Login incorrect' content,display a message `Telnet login incorrect'  with reference to arachmids  external attack identification systems.

\noindent \textit{Alert icmp any any $\to$ any any ( msg:''ICMP Source Quench''; itype: 4; icode:0;) }For an icmp packet, from any port of any address to a destination of any port on any address, with `itype:4'  which means ICMP source Quench,it is described as `no code ' or  ``code 0'' and for the `'icode:0 , this shoud create an an alert with the message `ICMP Source Quench'.

\noindent 

\noindent 
\paragraph{4.4 OpenSSL Heartbleed Vulnerability }

\noindent alert tcp \$EXTERNAL\_NET any -$>$ \$HOME\_NET 443 

\noindent msg:"SERVEROTHER OpenSSL TLSv1.2 heartbeat read overrun attempt" flow:to\_server,established content:"{\textbar}18 03 03{\textbar}"

\noindent depth:3 dsize:$>$40

\noindent detection\_filter:track by\_src, count 3, seconds 1

\noindent metadata:policy balanced-ips drop, policy security-ips drop, service ssl reference:cve,2014-0160 classtype:attempted-recon sid:30513 rev:2

\noindent \textit{Description and analysis:}

\begin{enumerate}
\item  The above rule will create an alert for some tcp packet leaving the external\_net on any port to the home\_net on port 443, .

\item  A message should flow the alert saying "SERVEROTHER OpenSSL TLSv1.2 heartbeat read overrun attempt".

\item  The flow option, helps to verify this is traffic going to the server on an established session

\item  The class type: attempted-recon,categorize a rule as detecting an intrusion or an attack that is known to be part of a more general type of attack class.

\item  The sid `30513' is normally used with rev, the sid acts as a unique snort rule identifier, it enables output plugins to identify rules easily.

\item  Rev is 2, it serves as an identifier for revisions on snort rules,it helps signatures and other description informations to be updated or replaced.

\item  The meta data allow the person writing the rules to add or embedd additional information in this case metadata:policy balanced-ips drop, policy security-ips drop, service ssl.

\item  The content:"{\textbar}18 03 03{\textbar}'', this is to find the 3 bytes data patterns "{\textbar}18 03 03{\textbar}"inside a packet eg.ASCII string or as binary data.,in this case ,the first three bytes ,"{\textbar}18 03 03{\textbar}" is meant to raise an alert if found in a packet.

\item  reference: cve,2014-0160, references are made to an external attack identification systems, eg. cve,2014-0160.

\item  The depth is 3, this enables the one writing the rule to specify how far a packet rule writer to specify how far into a packet, snort should search.

\item  The dsize:$>$40, this helps to test the packet payload size.

\item  Detection\_filter is the rate which must be exceeded by a party (source or destination
\end{enumerate}

\noindent ) before a rule can put out an event, counts are 3, which is the number matching rules in s seconds that will make an event filter limit to be exceeded. 1 seconds is the time over which count is accrued. ``track by\_dst'' is the rate which is tracked by the source IP address.

\noindent 

\noindent alert tcp \$HOME\_NET 443 -$>$ \$EXTERNAL\_NET any

\noindent msg:"SERVER-OTHER `TLSv1 large heartbeat response -- possible ssl Heartbleed attempt" flow:to\_client,established content:"{\textbar}18 03 01{\textbar}"

\noindent depth:3

\noindent byte\_test:2,$>$,128,0, relative 

\noindent detection\_filter:track by\_dst,count 5, seconds 60

\noindent metadata:,policy balanced-ips drop, policy security-ips drop, service ssl reference:cve,2014-0160 classtype:attempted-recon 

\noindent sid:30515 rev:3

\noindent \textit{Description and analysis:}

\begin{enumerate}
\item  The above rule will create an alert for some tcp packet leaving a home\_net on port 
\end{enumerate}

\noindent 443 to any port on the external\_net.

\begin{enumerate}
\item  A message should flow the alert saying "SERVER-OTHER `TLSv1 large heartbeat response -- possible ssl Heartbleed attempt"

\item  The flow option, helps to verify this is traffic going to the server on an established session

\item  The class type: attempted-recon, categorize a rule as detecting an intrusion or an attack that is known to be part of a more general type of attack class.

\item  The sid `30515' is normally used with rev, the sid acts as a unique snort rule identifier, it enables output plugins to identify rules easily.

\item  Rev is 3, it serves as an identifier for revisions on snort rules,it helps signatures and other description information to be updated or replaced 

\item  The meta data allow the person writing the rules to add or embedd additional information in this case metadata:policy balanced-ips drop, policy security-ips drop, service ssl.

\item  The content:"{\textbar}18 03 01{\textbar}'', this is to find the 3 bytes data patterns "{\textbar}18 03 03{\textbar}"inside a packet eg.ASCII string or as binary data.,in this case ,the first three bytes ,"{\textbar}18 03 03{\textbar}" is meant to raise an alert if found in a packet.

\item  reference: cve,2014-0160, references are made to an external attack identification systems, eg. cve,2014-0160

\item  Detection\_filter is the rate which must be exceeded by a party (source or destination) before a rule can put out an event,counts is 5,which is the number matching rules in s seconds that will make an event filterlimit to be exceeded. 60 seconds is the time period over which count is accrued.''track by\_dst'' is the rate which is tracked by the  destination IP address.

\item  The byte\_test argument is used in addition with an operator to test against a specific value,in this case ,byte\_test:2,$>$,128,0,relative.
\end{enumerate}

\noindent 
\paragraph{4.6 A Computer Worm}

\noindent The snort rule to detect the worm is:

\noindent \textit{alert tcp \$EXTERNAL\_NET any -$>$ \$HOME\_NET 1045 (msg:'' Internet Worm to be stopped''; sid:1000006; content: ``07''; rawbytes; within:1; content:''71 f2 03 01 04 9b 71 f2 01''; rawbytes;distance:0; content:''tire'';distance:-100;)}


\end{document}

